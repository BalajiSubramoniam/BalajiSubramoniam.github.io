\documentclass[oneside,11pt]{amsart}
\usepackage{capt-of}
\usepackage{xcolor}
\usepackage{fullpage}
\usepackage{amsmath}
\usepackage{amsfonts}
\usepackage{comment}
\usepackage{quiver}
\usepackage{amssymb}
\usepackage{amsthm}
\usepackage{graphicx}
\usepackage{tikz-cd}
\usepackage{enumitem}
\usepackage{nicefrac}
\usepackage{hyperref}
\usepackage[capitalise]{cleveref}
\usepackage[mathscr]{euscript}
\usepackage{mlmodern}
\usepackage[T1]{fontenc}
\usepackage[]{mdframed}

\newtheorem{thm}{Theorem}[section]
\crefname{thm}{Theorem}{Theorems}
\newtheorem{prop}[thm]{Proposition}
\crefname{prop}{Proposition}{Propositions}
\newtheorem{cor}[thm]{Corollary}

\newtheorem{lem}[thm]{Lemma}
\crefname{cor}{Corollary}{Corollaries}
\theoremstyle{definition}
\newtheorem{const}[thm]{Construction}
\newtheorem{ex}[thm]{Example}
\crefname{ex}{Example}{Examples}
\newtheorem{exer}[thm]{Exercise}
\crefname{ex}{Exercise}{Exercises}
\newtheorem{defn}[thm]{Definition}
\crefname{defn}{Definition}{Definitions}
\newtheorem{conj}[thm]{Conjecture}
\crefname{conj}{Conjecture}{Conjectures}

\newtheorem{obs}[thm]{Observation}
\newtheorem{ques}{Question}
\newtheorem{fact}[thm]{Fact}
\crefname{obs}{Observation}{Observations}

\newtheorem{notn}[thm]{Notation}
\crefname{notn}{Notation}{Notation}
\newtheorem{conv}[thm]{Convention}
\crefname{conv}{Convention}{Conventions}
\theoremstyle{remark}
\newtheorem{rem}[thm]{Remark}
\crefname{rem}{Remark}{Remarks}



\begin{document}
\title{AIS - Cohomology of Commutative Algebras}
\date{\today}
\maketitle




\begin{conv}
 Throughout, ``ring'' means a commutative ring with unity. If $R$ is a ring, then the category of $R$-modules is denoted by $R$-$\textsf{Mod}$.
\end{conv}



\section{K\"{a}hler differentials} 
\begin{defn}
	Let $k$ be a ring and $R$ a $k$-algebra. Let $M\in R$-\textsf{Mod}. A $k$-\emph{derivation} of $R$ in $M$ is a $k$-linear map $d:R \rightarrow M $ such that $d(ab)=ad(b)+bd(a)$ for all $a,b\in R$.           
\end{defn}
\begin{notn}
For an $R$-module $M$.   $\text{Der}_{k}(R,M)$ is the collection of $k$-derivations of $R$ in $M$. If $M=R$, we will write $\text{Der}_{k}(R)$ for $\text{Der}_{k}(R,R)$.      
\end{notn}
\begin{ex}
 \begin{enumerate}
	 \item Let $U\subseteq \mathbb{R} ^{n}$ be open and $x_{1},\ldots ,x_{n}$ be coordinates. Let $R=C^{\infty }(U)$ be the ring of smooth functions. Have derivations $\frac{\partial }{\partial x_{i}}\in \text{Der}_{\mathbb{R} }(U)$.
	 \item Let $R$ be the ring of smooth functions again. For $x\in U$, define $\mathfrak{m}_{x}$ to be the maximal ideal at $x$. Then, the maps $d_{i}:R \rightarrow \mathbb{R} =R/\mathfrak{m}_{x}$, $f \mapsto \frac{\partial f}{\partial x_{i}}(x)$ are $\mathbb{R} $-derivations of $R$ in $\mathbb{R} $.  
	 \item Formal derivatives in polynomial rings.
	 \item Let $R=k[x_{1},\ldots ,x_{n}]$ and $M=\bigoplus _{i=1}^{n} R dx_{i}$, where $dx_{i}$ are symbols. Define $d:R\rightarrow M$, $f\mapsto \sum_{i=1}^{n} \frac{\partial f}{\partial x_{i}} dx_{i}$. This is a derivation of $R$ in $M$.
	 \item Let $R$ be a $k$-algebra. Consider the map $\mu :R \otimes_{k} R \rightarrow R$, $a \otimes b \mapsto ab$. Let $I=\text{ker}(\mu )$. Exercise: Show that this kernel is generated by $\langle a \otimes 1 - 1\otimes a | a\in R \rangle$. Note that $R\otimes_{k} R$ has two $R$-module structures (multiplication on the left and on the right). Check that these two structures coincide on $I/I^{2}$ (use the fact that $(r\otimes 1 - 1\otimes r)(a\otimes 1 - 1 \otimes a)\in I^{2}$). The map $\delta : R \rightarrow I/I^{2} $, $a\mapsto (a\otimes 1 - 1\otimes a)$ is a $k$-derivation.        
 \end{enumerate}
 
\end{ex}

\begin{defn}
 For a $k$-algebra $R$, let $F$ be the free $R$-module generated by the set $\{ dr| r\in R \} $. Let $N$ be the submodule generated by $\langle d(rr')-rdr'-r'dr,\; d(ar+a'r')-adr-a'dr' | r,r'\in R  \rangle  \subseteq F$. Let $\Omega _{R/k}:=F/N$. By definition, the map $\phi :R \rightarrow \Omega _{R/k}$, $r \mapsto dr$ is a derivation. In fact, for any $M\in R$-\textsf{Mod}, there is a natural isomorphism  $\text{Der}_{k}(R,M)\cong \text{Hom}_{R\text{-}\textsf{Mod} }(\Omega_{R/k},M)$.
\end{defn}

\begin{rem}
 The functor $\text{Der}_{k}(R,\_):R\text{-}\textsf{Mod} \rightarrow \textsf{Set}$ is represented by $\Omega _{R/k}$.  
\end{rem}

\begin{exer}
 The object (in our case $\Omega _{R/k}$) representing a functor is unique up to unique isomorphism.
\end{exer}

\begin{ex}
	Let $R=k[x_{1},\ldots ,x_{n}]$. Then, $\Omega _{R/k}$ is generated by $dx_{1},\ldots ,dx_{n}$. We claim that $dx_{1},\ldots ,dx_{n}$ are also linearly independent. That is, $\Omega _{R/k}=\bigoplus Rdx_{i}$. Consider the free $R$-module $R^{\oplus n}$ with basis $e_{1},\ldots ,e_{n}$. To see this, use the isomorphism $\text{Der}_{k}(R,M)\cong \text{Hom}_{R\text{-}\textsf{Mod} }(\Omega_{R/k},M)$ for when $M=R^{\oplus n}$ and the differential $\delta : R^{\oplus n} \rightarrow R$, $f\mapsto \sum_{}^{} \frac{\partial f}{\partial x_{i}} e_{i}$. 
\end{ex}

\begin{notn}
 $R\ltimes M$ is the $R$-module $R\oplus M$ with multiplication defined as $(r,m)\cdot (r',m')=(rr', rm'+r'm)$.    
\end{notn}

\begin{prop}
	Let $I=\emph{ker}(R\otimes_{k} R \rightarrow R)$ and $\delta :R \rightarrow I/I^{2} $, $r\mapsto r\otimes 1 - 1\otimes r$. Then, for $M\in R$-\emph{\textsf{Mod}}, for all $\phi \in \emph{Der}_{k}(R,M)$, there exists a unique $R$-linear map $\tilde{\phi }:I/I^{2} \rightarrow  M$ such that $e=\tilde{e}\circ \delta $. In particular, there is a unique isomorphism $(I/I^{2},\delta )\cong (\Omega _{R/k},d)$.     
\end{prop}


\begin{proof}
	The first statement says that $I/I^{2}$ represents the functor $\text{Der}_{k}(R,\_)$. As representing objects are unique up to unique isomorphism, the second statement follows. For the first statement, consider $\phi \in \text{Der}_{k}(R,M)$. Define $\check{\phi }: R \rightarrow R \ltimes M$, $r \mapsto (r,\phi (r))$, which is a $k$-algebra homomorphism. There is also an inclusion homomorphism $i:R \rightarrow R\ltimes M $, $r \mapsto (r,0)$. Thus, we have a canonical $k$-algebra homomorphism $h: R \otimes _{k} R \rightarrow R \ltimes M$, $r\otimes r' \mapsto \check{\phi }(r)i(r')=(rr', r'e(r))$. Note that $h(r\otimes 1 - 1\otimes r)=(0,e(r))$. Thus, $h(I^{2})=0$ and $h$ descends to a map $\tilde{\phi }:I/I^{2} \rightarrow M$.       
\end{proof}

\begin{thm}[First fundamental sequence]
 Let $k \rightarrow R \rightarrow S$ be maps of rings. Then, there is an exact sequence of $S$-modules
 \begin{equation*}
	 S\otimes _{R} \Omega_{R/k} \xrightarrow{\alpha } \Omega _{S/k} \xrightarrow{\beta } \Omega _{S/R} \rightarrow 0
 \end{equation*}
 where $\alpha (s\otimes d_{R/k}r)=sd_{S/k}r$ and $\beta (d_{S/k}(s))=d_{S/R}(s)$  
\end{thm}
\begin{proof}
 Clearly, $\beta $ is a surjection (same generators) with kernel (exercise: check this) $\langle \{ d_{S/k}r | r\in R \}  \rangle $. This is clearly the image of $\alpha $.  
\end{proof}

\begin{thm}[Second Fundamental Sequence]
 As before, $R$ is a $k$-algebra. Let $I\subseteq R$ be an ideal and $S=R/I$. There is an exact sequence of $S$-modules
 \begin{equation*}
	 I/I^{2} \xrightarrow{\gamma  } S \otimes_{R} \Omega _{R/k} \xrightarrow{\gamma '} \Omega _{S/k} \rightarrow 0 
 \end{equation*}
 where $\gamma  :r\mapsto 1 \otimes d_{R/k}r$ and $\gamma ' : s\otimes d_{R/k}r \mapsto sd_{S/k}r$  
\end{thm}
\begin{proof}
 Note that $\gamma '$ is the same as $\alpha $. As $R\rightarrow S$ is surjective, $\Omega _{S/R}=0$. We claim (proof left as an exercise) that $\text{ker}(\Omega _{R/k}\rightarrow \Omega _{S/k})=I\Omega _{R/k}+R\{ d_{R/k}r|r\in I \} $. Assuming this, we see 
 \begin{equation*}
  \text{ker}(S\otimes _{R} \Omega _{R/k} \rightarrow \Omega _{S/R})=\langle 1\otimes d_{R/k}r|r\in I \rangle = \text{Im}(\gamma )  
 \end{equation*}
 Check that $\gamma $ is $S$-linear.  
\end{proof}

\subsection*{Regularity} 

\begin{prop}
 Let $k$ be a field and $(R,\mathfrak{m} )$ a local $k$-algebra such that $K=R/\mathfrak{m} $ is a finite separable extension of $k$ and $\mathfrak{m} $ has finite embedding dimension. Then, $\Omega _{K/k}=0$. 
\end{prop}
\begin{proof}
	Let $a\in K$ with $f\in k[x]$ being the minimal polynomial of $a$ over $k$. Then, for any $k$-derivation, $0=df(a)=f'(a)da$. As $K/k$ is separable, $f'(a)\neq 0$ which means that $da=0$. From the second fundamental sequence, we get a $K$-linear surjection
	\begin{equation*}
		\mathfrak{m} /\mathfrak{m}^{2} \xrightarrow{\delta } K \otimes _{R} \Omega _{R/k} \rightarrow \Omega _{K/k} 
	\end{equation*}
We claim that $\delta $ is an isomorphism. If true, we conclude that $\text{dim}_{K}(\mathfrak{m} /\mathfrak{m}^{2} )=\text{dim}_{K}(K\otimes _{R} \Omega _{R/k})\geqslant \text{dim}(R)$. To prove this, it suffices to show that 
\begin{equation*}
 \delta ^{\vee} :(K\otimes _{R} \Omega _{R/k})^{\vee} \rightarrow (\mathfrak{m} /\mathfrak{m}^{2} )^{\vee}
\end{equation*}
is surjective. Let $f:\mathfrak{m} /\mathfrak{m} ^{2} \rightarrow K $. Extend $f$ to $R$ using any $k$-linear splitting (by composition with $R \twoheadrightarrow R/\mathfrak{m}^{2}$)
\begin{equation*}
 0 \rightarrow \mathfrak{m}/\mathfrak{m}^{2} \rightarrow R/\mathfrak{m}^{2} \rightarrow K \rightarrow 0 
\end{equation*}
Set $\tilde{f}(K)=0$. Then (exercise) $\tilde{f}\in \text{Der}_{k}(R,K)$. Let $g\in \text{Hom}_{R}(\Omega _{R/k},K)$ be the corresponding linear map. Then, since $\text{Hom}_{R}(\Omega _{R/k},K)=\text{Hom}_{R}(K\otimes _{R} \Omega _{R/k},K)=\text{Hom}_{K}(K\otimes _{R} \Omega _{R/k},k)$, for all $r\in \mathfrak{m} $, $f(r)=\tilde{f}(r)=g(d_{R/k}(r))$. So, $f=g\delta $. 
\end{proof}





\newpage
\section{Smoothness} 
\begin{notn}
 Let $k$ be a commutative ring and $R$ is a commutative $k$-algebra. 
\end{notn}
\begin{defn}
	Let $M\in R\textsf{-Mod}$. A \emph{square-zero extension} of $R$ by $M$ is a short exact sequence of $k$-modules 
 \begin{equation*}
	 0 \rightarrow M\xrightarrow{\theta } E \rightarrow R \rightarrow 0
 \end{equation*}
 where $E$ is a $k$-algebra, $E \rightarrow R$ is a $k$-algebra morphism and $\theta(M)\subseteq E$ is an ideal such that $\theta (M)^{2}=0$.  
\end{defn}
\begin{rem}
 In the above case, identify $M$ with $\theta(M)$, an ideal in $E$. Hence, $E$ acts $k$-linearly on $M$. So, $M^{2}=0$ essentially means that this action factors as follows.
 % https://q.uiver.app/#q=WzAsMyxbMCwwLCJFIl0sWzIsMCwiXFx0ZXh0e0VuZH1fayhNKSJdLFsxLDEsIkUvTT1SIl0sWzAsMV0sWzAsMl0sWzIsMSwiIiwyLHsic3R5bGUiOnsiYm9keSI6eyJuYW1lIjoiZGFzaGVkIn19fV1d
\[\begin{tikzcd}
	E && {\text{End}_k(M)} \\
	& {E/M=R}
	\arrow[from=1-1, to=1-3]
	\arrow[from=1-1, to=2-2]
	\arrow[dashed, from=2-2, to=1-3]
\end{tikzcd}\]
\end{rem}

\begin{ex} For a field $k$, 
 \begin{equation*}
	 0 \rightarrow k\equiv k\varepsilon  \rightarrow k[\varepsilon ]/(\varepsilon ^{2}) \rightarrow k \rightarrow 0
 \end{equation*}
 is a square-zero extension of $k$ by itself.  
\end{ex}

\begin{defn}
 An extension $E$  of $R$ by $M$
 \begin{equation*}
  0 \rightarrow M \rightarrow E \rightarrow R \rightarrow 0
 \end{equation*}
 is \emph{trivial }if it is isomorphic to the extension
 \begin{equation*}
  0 \rightarrow M \rightarrow R \ltimes M \rightarrow R \rightarrow 0
 \end{equation*}
 That is, we have
 % https://q.uiver.app/#q=WzAsMTAsWzAsMCwiMCJdLFsxLDAsIk0iXSxbMiwwLCJFIl0sWzMsMCwiUiJdLFs0LDAsIjAiXSxbMCwyLCIwIl0sWzEsMiwiTSJdLFsyLDIsIk1cXGx0aW1lcyBSIl0sWzMsMiwiUiJdLFs0LDIsIjAiXSxbMCwxXSxbMSwyXSxbMiwzXSxbMyw0XSxbNSw2XSxbNiw3XSxbNyw4XSxbOCw5XSxbMyw4LCIiLDEseyJsZXZlbCI6Miwic3R5bGUiOnsiaGVhZCI6eyJuYW1lIjoibm9uZSJ9fX1dLFsxLDYsIiIsMSx7ImxldmVsIjoyLCJzdHlsZSI6eyJoZWFkIjp7Im5hbWUiOiJub25lIn19fV0sWzIsNywiXFx0ZXh0e2stYWxnIGlzb20ufSIsMV1d
\[\begin{tikzcd}
	0 & M & E & R & 0 \\
	\\
	0 & M & {M\ltimes R} & R & 0
	\arrow[from=1-1, to=1-2]
	\arrow[from=1-2, to=1-3]
	\arrow[Rightarrow, no head, from=1-2, to=3-2]
	\arrow[from=1-3, to=1-4]
	\arrow["{\text{k-alg isom.}}"{description}, from=1-3, to=3-3]
	\arrow[from=1-4, to=1-5]
	\arrow[Rightarrow, no head, from=1-4, to=3-4]
	\arrow[from=3-1, to=3-2]
	\arrow[from=3-2, to=3-3]
	\arrow[from=3-3, to=3-4]
	\arrow[from=3-4, to=3-5]
\end{tikzcd}\]
\end{defn}

\begin{obs}
 An extension 
 \begin{equation*}
	 0 \rightarrow M \xrightarrow{\alpha } E  \xrightarrow{\beta } R \rightarrow 0
 \end{equation*}
 is trivial if and only if there is a $k$-algebra section $\sigma :R \rightarrow E $ of $\beta $. Check that $E\cong R \ltimes M$ by writing $e=\sigma \beta (e)- (\sigma \beta (e)-e)$, where the first term belongs to $R$ and the second to $M$.      
\end{obs}
\begin{defn}
 Let $R$ be a $k$-algebra. We say that $R$ is \emph{smooth} over $k$ if for every square-zero extension $$0 \rightarrow M \rightarrow E \rightarrow T \rightarrow 0$$
 of commutative $k$-algebras and every $k$-algebra map $u:R \rightarrow T $, there exists a $k$-algebra lifting $v:R \rightarrow E $ of $u$.      
\end{defn}
\begin{ex}
	$k[x_{1},\ldots ,x_{n}]$ is free over $k$ and is hence smooth.
\end{ex}

\begin{rem}
 Let $R$ be a smooth $k$-algebra and 
 \begin{equation}\label{sm21}
  0 \rightarrow J \rightarrow E \rightarrow R \rightarrow 0
 \end{equation}
 a nilpotent extension. That is, $J^{m}=0$ for some $m\geqslant 1$. Then, smoothness implies that \cref{sm21} is split. The idea is to induct using the following diagram
 \begin{equation*}
  0 \rightarrow {J^{n}/J^{n+1}} \rightarrow E/J^{n+1} \rightarrow E/J^{n} \rightarrow 0
 \end{equation*}
 which is a square-zero extension. 
\end{rem}

\begin{prop}
	Let $k \rightarrow R \xrightarrow{f} S$ be maps of rings 
	\begin{enumerate}[label=\emph{\normalfont{(\arabic*)}}]
 \item If $S$ is smooth over $R$, then the first fundamental exact sequence 
	 \begin{equation*}
		 0 \rightarrow \Omega _{R/k}\otimes_{R} S \xrightarrow{\alpha } \Omega _{S/k} \rightarrow \Omega _{S/R} \rightarrow 0
	 \end{equation*}
	is split exact.
 \item If $S=R/I$ and suppose  $S$ is smooth over $k$, then the second fundamental exact sequence is split exact
	 \begin{equation*}
	  0 \rightarrow I/I^{2} \rightarrow \Omega _{R/k} \otimes _{R} S \rightarrow \Omega _{S/k} \rightarrow 0
	 \end{equation*}
	 
\end{enumerate}

\end{prop}
\begin{proof}
 As $\text{Hom}_{S}(\Omega _{S/k}, \Omega _{R/k} \otimes _{R} S)\cong \text{Der}_{k}(S,\Omega _{R/k}\otimes _{R} S)$, for a splitting, we need a suitable derivation. Before proceeding, we introduce a general construction.

 Let $N\in S\textsf{-Mod}$ and $\delta \in \text{Der}_{k}(R,N)$. This defines a $k$-algebra map $\phi :R \rightarrow S\ltimes N $, $r \mapsto (f(r), \delta r)$. Since $S$ is smooth over $R$, the square-zero extension of $R$-algebras 
 \begin{equation*}
	 0 \rightarrow N \xrightarrow{i_{2}} S \ltimes N \xrightarrow{\pi _{1}} S \rightarrow 0
 \end{equation*}
 has an $R$-algebra splitting $\sigma $. Note that $\sigma \circ f = \phi $ as $\sigma $ is an $R$-algebra morphism. Let $\delta '= \pi _{2}\sigma \in \text{Der}_{R}(S,N)$, where $\pi _{2}: S \ltimes N \rightarrow  N$ is the projection to the second coordinate. Check that $\delta 'f = \delta $. Now set $N= \Omega _{R/k} \otimes _{R} S$ and $\delta = d_{R/k}\otimes 1$. Let $\gamma \in \text{Hom}_{S}(\Omega _{S/k}, \Omega _{R/k}\otimes  _{R}S)$ be the map corresponding to $\delta '$. That is, $\delta '=\gamma d_{S/k}$. Then, for $r\in R$ and $s\in S$,
 \begin{equation*}
	 d_{R/k}r\otimes s \overset{\alpha }{\mapsto } sd_{S/k}(f(r)) \overset{\gamma }{\mapsto } s(d_{R/k}r \otimes 1) = d_{R/k}r \otimes s 
 \end{equation*}
 Thus, $\gamma $ is a splitting. Proof of the second statement is left as an exercise.
\end{proof}
\begin{prop}
 Let $R$ be a smooth $k$-algebra, then $\Omega _{R/k}$ is projective.
\end{prop}
\begin{proof}
 Note that every map $f:M \rightarrow N $ of $R$-modules induces a map of $R$-algebras 
 \begin{equation*}
	 R \ltimes M \xrightarrow{\text{Id},f} R \ltimes N 
 \end{equation*}
 Conversely, every map of $R$-algebras $\phi : R \ltimes M  \rightarrow R \ltimes N$ yields an $R$-linear map $\phi |_{0\oplus M} :M \rightarrow N $. Let $I=\text{ker}(R^{e}:=R\otimes_{k} R \rightarrow R)$. Then, $I/I^{2}=\Omega _{R/k}$. As $R/k$ is smooth, the extension 
 \begin{equation*}
	 0 \rightarrow \Omega _{R/k} \rightarrow R^{e}/I^{2} \xrightarrow{\mu } R \rightarrow 0
 \end{equation*}
 is trivial with an isomorphism $\gamma : R^{e}/I^{2} \rightarrow R \ltimes \Omega _{R/k}$ . Let $p: R^{e} \rightarrow R^{e}/I^{2} $ be the surjection. Give $R^{e}=R\otimes_{k} R$ the $R$-module structure given by left multiplication. Then, $p, \mu , \gamma $  are all $R$-algebra homomorphisms. Let $M,N\in R\textsf{-Mod} $ and consider the diagram  
% https://q.uiver.app/#q=WzAsNCxbMCwyLCJNIl0sWzIsMiwiTiJdLFs0LDIsIjAiXSxbMiwwLCJcXE9tZWdhX3tSL2t9Il0sWzAsMV0sWzEsMl0sWzMsMV1d
\[\begin{tikzcd}
	&& {\Omega_{R/k}} \\
	\\
	M && N && 0
	\arrow["h",from=1-3, to=3-3]
	\arrow[from=3-1, to=3-3]
	\arrow[from=3-3, to=3-5]
\end{tikzcd}\]
of $R$-module homomorphisms. This gives a diagram of $R$-algebras

% https://q.uiver.app/#q=WzAsOSxbNCwwLCJSXmUiXSxbNCwyLCJSXFxsdGltZXMgTSJdLFs2LDIsIlJcXGx0aW1lcyBOIl0sWzMsMCwiUlxcb3RpbWVzX2sgUiJdLFsyLDAsIkkgIl0sWzYsMCwiUiBcXGx0aW1lcyBcXE9tZWdhX3tSL2t9Il0sWzgsMiwiMCJdLFsyLDIsIlxcdGV4dHtrZXJ9KDEsZikiXSxbMCwyLCIwIl0sWzAsMSwiXFxleGlzdHMgcCIsMix7InN0eWxlIjp7ImJvZHkiOnsibmFtZSI6ImRhc2hlZCJ9fX1dLFsxLDIsIigxLGYpIl0sWzMsMCwiXFxjb25nIiwxLHsic3R5bGUiOnsiYm9keSI6eyJuYW1lIjoibm9uZSJ9LCJoZWFkIjp7Im5hbWUiOiJub25lIn19fV0sWzQsMywiXFxzdWJzZXRlcSIsMSx7InN0eWxlIjp7ImJvZHkiOnsibmFtZSI6Im5vbmUifSwiaGVhZCI6eyJuYW1lIjoibm9uZSJ9fX1dLFswLDVdLFs1LDIsIigxLGgpIl0sWzIsNl0sWzcsMV0sWzgsN11d
\[\begin{tikzcd}
	&& {I } & {R\otimes_k R} & {R^e} && {R \ltimes \Omega_{R/k}} \\
	\\
	0 && {\text{ker}(1,f)} && {R\ltimes M} && {R\ltimes N} && 0
	\arrow["\subseteq"{description}, draw=none, from=1-3, to=1-4]
	\arrow["\cong"{description}, draw=none, from=1-4, to=1-5]
	\arrow["\gamma p",from=1-5, to=1-7]
	\arrow["{\exists \phi }"', dashed, from=1-5, to=3-5]
	\arrow["{(1,h)}", from=1-7, to=3-7]
	\arrow[from=3-1, to=3-3]
	\arrow[from=3-3, to=3-5]
	\arrow["{(1,f)}", from=3-5, to=3-7]
	\arrow[from=3-7, to=3-9]
\end{tikzcd}\]
The bottom extension is square-zero. As an exercise, prove that $R^{e}=R\otimes _{k} R$ is smooth over $R$. So, there exists $\phi $ as in the diagram so that it commutes. By the commutativity of the above diagram, $\gamma p (I)\subseteq 0 \oplus \Omega _{R/k}$. So, $((1,h)\circ \gamma p) (I) \subseteq 0 \oplus M$. Thus, we get an $R$-algebra map $\tilde{\phi }: R \ltimes \Omega _{R/k} \rightarrow  R \ltimes M$ with $\text{Im}(\Omega _{R/k})\subseteq 0 \oplus M$. This gives a lifting of $h$ to an $R$-linear map $\pi _{2}\circ \tilde{\phi }\circ i_{2}:\Omega _{R/k} \rightarrow M $     
\end{proof}



We'll prove the following result next lecture.
\begin{prop}
 If $R$ is a noetherian local ring containing a field $k$  that is smooth over $k$, then $R$ is a regular local ring.
\end{prop}










\newpage
\section{More on Smoothness \& Prelude to Andr\'e-Quillen} 
\begin{prop}
 Let $(R,\mathfrak{m} )$ be a noetherian local containing a field. If $R$ is smooth over $k$, then $R$ is a regular local ring. 
\end{prop}
\begin{proof}
	We may assume that $k$ is a prime field. Any bigger field is smooth over the prime field. This ensures that $K=R/\mathfrak{m} $ is smooth over $k$. Let $d=\text{dim}_{K}(\mathfrak{m}/\mathfrak{m}^{2} )$. Let $S=K[x_{1},\ldots ,x_{d}]_{(x_{1},\ldots ,x_{d})}$ and $M\subseteq S$ the maximal ideal. Since $K/K$ is smooth, the following square-zero extension of $K$-algebras is trivial. 
\begin{equation*}
 0 \rightarrow M/M^{2} \rightarrow S/M^{2} \rightarrow K \rightarrow 0
\end{equation*}


	So, $S/M^{2} \cong K \ltimes M/M^{2} $ as $K$-algebras. As $K/k$ is smooth, similarly, $R/\mathfrak{m}^{2}\cong K \ltimes \mathfrak{m}/\mathfrak{m}^{2}$ as $k$-algebras. Since $M/M^{2} \cong \mathfrak{m}/\mathfrak{m}^{2} $ as $K$-algebras, we see that $R/\mathfrak{m}^{2} \cong S/M^{2}$.  As $R$ is smooth over $k$, the surjection $R \twoheadrightarrow S/M^{2}$ can be lifted to a surjective (Nakayama) map $R \twoheadrightarrow S/M^{n}$ for all $n\geqslant 2$. 
	% https://q.uiver.app/#q=WzAsNixbMCwxLCIwIl0sWzEsMSwiXFxtYXRoZnJha3ttfV4yL1xcbWF0aGZyYWt7bX1eMyJdLFsyLDEsIlMvTV4zIl0sWzMsMSwiUy9NXjIiXSxbNCwxLCIwIl0sWzMsMCwiUiJdLFswLDFdLFsxLDJdLFsyLDNdLFszLDRdLFs1LDMsIiIsMCx7InN0eWxlIjp7ImhlYWQiOnsibmFtZSI6ImVwaSJ9fX1dLFs1LDIsIiIsMCx7InN0eWxlIjp7ImJvZHkiOnsibmFtZSI6ImRhc2hlZCJ9LCJoZWFkIjp7Im5hbWUiOiJlcGkifX19XV0=
\[\begin{tikzcd}
	&&& R \\
	0 & {\mathfrak{m}^2/\mathfrak{m}^3} & {S/M^3} & {S/M^2} & 0
	\arrow[dashed, two heads, from=1-4, to=2-3]
	\arrow[two heads, from=1-4, to=2-4]
	\arrow[from=2-1, to=2-2]
	\arrow[from=2-2, to=2-3]
	\arrow[from=2-3, to=2-4]
	\arrow[from=2-4, to=2-5]
\end{tikzcd}\]
We claim that $\text{ker}(R \rightarrow S/M^{n})=\mathfrak{m}^{n}$. In fact it is sufficient to see that $\mathfrak{m}^{n}$ is contained in $\text{ker}(R \rightarrow S/M^{n})=\mathfrak{m}^{n}$.

Let $H_{R}$ and $H_{S}$ be the Hilbert Samuel polynomials of $R$ and $S$ respectively. Recall that 
\begin{enumerate}
 \item $H_{R}(n)=\text{length}(R/\mathfrak{m}^{n})$ and $H_{S}(n)=\text{length}(S/M^{n})$.
 \item $\text{deg}(H_{R})=\text{dim}(R)$ and $\text{deg}(H_{S})=\text{dim}(S)=d$   
\end{enumerate}
Then, for all $n\geqslant 2$, $H_{R}(n)\geqslant H_{S}(n)$. Hence, $\text{dim}(R)\geqslant d$. However, we know that $\text{dim}(R)\leqslant d=\text{dim}_{K}(\mathfrak{m}/\mathfrak{m}^{2} )$. Thus, $\text{dim}(R)=\text{dim}_{K}(\mathfrak{m}/\mathfrak{m}^{2} )$. 
\end{proof}
\begin{defn}
 Let $k$ be a field and $R$ be a $k$-algebra. $R$ is said to be \emph{geometrically regular} if for every finite extension $k \hookrightarrow l$, the ring $R \otimes _{k}l$ is regular.   
\end{defn}
\begin{fact}
 A $k$-algebra $R$ is smooth if and only if $R$ is geometrically regular. 
\end{fact}

We now turn to the construction and properties of the $T^{i}$-functors (following Hartshorne's Deformation Theory).  
\begin{const}
 Let $A$ be a ring and $B$ an $A$-algebra. Let $e_{0}: R \rightarrow B$ be a surjective $A$-algebra homomorphism, where $A$ is a polynomial ring over $A$ (we can always choose such $e_{0}$). Let $I=\text{ker}(e_{0})$ and let \begin{equation*}
	 0 \rightarrow Q \xrightarrow{\alpha } F \xrightarrow{j} I \subseteq R \rightarrow 0 
 \end{equation*}
      be an exact sequence of $R$-modules such that $F$ is free over $R$. More elaborately, you pick a generating set of $I$, a sufficiently large free module $F$ over the polynomial ring, construct a surjective $R$-module map $j$ and define $\alpha $ to be the kernel of $j$. Let (you should be reminded of the Koszul complex here)
      \begin{equation*}
       F_{0}=\langle \{ j(x)y-xj(y)|x,y\in F \}  \rangle \subseteq F 
      \end{equation*}
We note that $F_{0}\subseteq Q$ is a submodule and $IQ\subseteq F_{0}$. We have a complex of $B$-modules      
\begin{equation}
\begin{tikzcd}
	& {L_2} && {L_1} && {L_0} \\
	{\mathbb{L}_{\bullet}:=} & {Q/F_0} && {(F/F_0)\otimes_R B} && {\Omega_{R/A} \otimes_R B} \\
	&&& {F/IF} && {I/I^2} \\
	{\text{Degree}:} & {(2)} && {(1)} && {(0)}
	\arrow["{:=}"{marking, allow upside down}, draw=none, from=1-2, to=2-2]
	\arrow["{:=}"{marking, allow upside down}, draw=none, from=1-4, to=2-4]
	\arrow["{:=}"{marking, allow upside down}, draw=none, from=1-6, to=2-6]
	\arrow["{\overline{\alpha}}", from=2-2, to=2-4]
	\arrow["{\delta\circ \overline{j}}", from=2-4, to=2-6]
	\arrow["\cong"{description}, draw=none, from=2-4, to=3-4]
	\arrow["{\overline{j}}"', from=3-4, to=3-6]
	\arrow["\delta"', from=3-6, to=2-6]
\end{tikzcd}
\end{equation}
Note that $L_{1},L_{0}$ are free $B$-modules. For any $M\in B$-\textsf{Mod}, define
\begin{equation*}
	T^{i}(B/A,M):=H^{i}(\text{Hom}(\mathbb{L}_{\bullet},M) ) 
\end{equation*}
for $i=0,1,2$. 
\end{const}

There are many unnatural choices we've made here but the following lemma says that these do not matter.

\begin{lem}
 The modules $T^{i}(B/A,M)$ are independent of the choices of $R$, $F$.  
\end{lem}
\begin{thm}\label{thma}
 Let $B$ be an $A$-algebra and $$0 \rightarrow M' \rightarrow M \rightarrow M''$$ be a short exact sequence of $B$-modules. Then, there is a long exact sequence 
 \begin{equation*}
  0 \rightarrow T^{0}(B/A,M') \rightarrow T^{0}(B/A,M) \rightarrow T^{0}(M/M'') \rightarrow T^{1}(B/A,M') \rightarrow \ldots \rightarrow T^{2}(B/A,M'')
 \end{equation*}
\end{thm}
\begin{proof}
 As $L_{0}$ and $L_{1}$ are free $B$-modules,
 \begin{equation*}
  0 \rightarrow \text{Hom}_{B}(L_\bullet,M') \rightarrow \text{Hom}_{B}(L_{\bullet},M) \rightarrow \text{Hom}_{B}(L_{\bullet},M'') \rightarrow 0   
 \end{equation*}
 is exact except for $i=2$ (where it is left exact). Easy calculation of homology. 
\end{proof}




\begin{thm}[Jacobi-Zariski]\label{thmb}
 Let $A \rightarrow B \rightarrow  C$ be homomorphisms of rings. Let $M\in C$\emph{-\textsf{Mod}}. Then, there is an exact sequence of $C$-modules
\begin{equation*}
 0 \rightarrow T^{0}(C/B,M) \rightarrow T^{0}(C/A,M) \rightarrow T^{0}(B/A,M) \rightarrow T^{1}(C/B,M) \rightarrow \ldots  \rightarrow T^{2}(C/A,M) \rightarrow T^{2}(B/A,M)
\end{equation*}

\end{thm}
\begin{proof}
 Omitted
\end{proof}
\begin{prop}
 For any map of rings $A \rightarrow B$ and $M\in B$\emph{-\textsf{Mod}}, $T^{0}(B/A,M)=\emph{Hom}(\Omega _{B/A},M)=\emph{Der}_{A}(B,M)$    
\end{prop}
\begin{proof}
 Write $B$ as a quotient
 \begin{equation*}
  0 \rightarrow I \rightarrow R \rightarrow B \rightarrow 0
 \end{equation*}
 where $R$ is a polynomial ring. Then, there is an exact sequence $$I/I^{2} \rightarrow \Omega _{R/A} \otimes _{A} B \rightarrow \Omega _{B/A} \rightarrow 0$$
 From the construction of $\mathbb{L}_{\bullet }$, we have a surjective map $L_{1} \rightarrow I/I^{2}$. Thus, the sequence $L_{1} \rightarrow L_{0} \rightarrow \Omega _{B/A} \rightarrow 0$ is exact. Taking $\text{Hom}(\mathbb{L}_{\bullet },M)$, we get $T^{0}(B/A,M)=\text{Hom}(\Omega _{B/A},M)$.     
\end{proof}
\begin{prop}\label{prop2}
 If $B$ is a polynomial ring over $A$, then $T^{i}(B/A,M)=0$ for $i=1,2$ for all $M$.   
\end{prop}
\begin{proof}
 Take $R=B$ in the construction. Then, $I=0$ and $F=0$. So, $L_{2}=L_{1}=0$.    
\end{proof}
\begin{prop}\label{prop3}
 If $A \twoheadrightarrow B $ is a surjective ring homomorphism with kernel $I$, then $T^{0}(B/A,M)=0$ for all $M$ and $T^{1}(B/A,M)=\emph{Hom}_{B}(I/I^{2},M)$.
\end{prop}
\begin{proof}
	The first statement follows from \cref{prop2} and Problem 2 of Tutorial 1 (which says that $S^{-1}\Omega _{R/k}\cong \Omega _{S^{-1}R/k})$. Take $R=A$. Then, $L_{0}=0$. Further, the exact sequence 
	\begin{equation*}
	 0 \rightarrow Q \rightarrow F \rightarrow E \rightarrow 0
	\end{equation*}
gives
% https://q.uiver.app/#q=WzAsNyxbMCwwLCJRXFxvdGltZXNfQSBCIl0sWzIsMCwiRlxcb3RpbWVzX0EgQiJdLFs0LDAsIkkvSV4yIl0sWzYsMCwiMCJdLFswLDEsIlEvSVEiXSxbMSwyLCJRL0ZfMD1MXzIiXSxbMiwxLCJGL0lGPUxfMSJdLFswLDFdLFsxLDJdLFsyLDNdLFswLDQsIiIsMix7ImxldmVsIjoyLCJzdHlsZSI6eyJoZWFkIjp7Im5hbWUiOiJub25lIn19fV0sWzQsNSwiIiwyLHsic3R5bGUiOnsiaGVhZCI6eyJuYW1lIjoiZXBpIn19fV0sWzUsNl0sWzEsNiwiIiwyLHsibGV2ZWwiOjIsInN0eWxlIjp7ImhlYWQiOnsibmFtZSI6Im5vbmUifX19XV0=
\[\begin{tikzcd}
	{Q\otimes_A B} && {F\otimes_A B} && {I/I^2} && 0 \\
	{Q/IQ} && {F/IF=L_1} \\
	& {Q/F_0=L_2}
	\arrow[from=1-1, to=1-3]
	\arrow[Rightarrow, no head, from=1-1, to=2-1]
	\arrow[from=1-3, to=1-5]
	\arrow[Rightarrow, no head, from=1-3, to=2-3]
	\arrow[from=1-5, to=1-7]
	\arrow[two heads, from=2-1, to=3-2]
	\arrow[from=3-2, to=2-3]
\end{tikzcd}\]
Taking $\text{Hom}_{B}(\_,M)$ and cohomology, we get 
\begin{equation*}
 T^{1}(B/A,M) = \text{Hom}_{B}(I/I^{2},M) 
\end{equation*}

\end{proof}
\begin{thm}
 Let $k$ be an algebraically closed field and $B$ be a finite type $k$-algebra. Then, $B$ is smooth over $k$ if and only if $T^{1}(B/k,M)=0$ for all $M$.    
\end{thm}
\begin{proof}
	Let $B=A/I$, where $A=k[x_{1},\ldots ,x_{n}]$. Then, $B$ is smooth over $k$ if and only if the conormal sequence 
	\begin{equation*}
	 0 \rightarrow I/I^{2} \rightarrow \Omega _{A/k} \otimes _{A} B \rightarrow \Omega _{B/k} \rightarrow 0
	\end{equation*}
	is split exact, and $\Omega _{B/k}$ is projective. We will prove the theorem modulo the following claim:\\
	There is an exact sequence
	\begin{equation*}
		0 \rightarrow T^{0}(B/k,M) \rightarrow \text{Hom}_{A}(\Omega _{A/k},M) \xrightarrow{\eta } \text{Hom}_{B}(I/I^{2},M) \rightarrow T^{1}(B/k,M) \rightarrow 0   
	\end{equation*}
	Note that $\text{Hom}_{A}(\Omega _{A/k},M)=\text{Hom}_{B}(\Omega _{A/k} \otimes _{A} B)$, and $\eta $ is the map $\text{Hom}_{B}(\Omega _{A/k} \otimes _{A} B, M) \rightarrow \text{Hom}_{B}(I/I^{2},M)$. Note that $\eta $ is surjective if and only if $T^{1}(B/k,M)=0$ for all $M$. Suppose that $B/k$ is smooth. Then, the conormal sequence is split exact and hence, $\eta $ is surjective.

	Conversely, suppose $T^{1}(B/k,M)=0$ for all $M$. By the claim, $\eta $ is surjective. Hence, there exists $\sigma \in \text{Hom}_{B}(\Omega _{A/k }\otimes _{A} B, M)$ such that $\eta (\sigma )=\text{Id}_{I/I^{2}}$. This gives a splitting of the conormal exact sequence. Moreover, by \cref{thma}, $T^{0}(B/k,\_)=\text{Hom}_{B}(\Omega _{B/k},\_)$ is an exact functor. Hence, $\Omega _{B/k}$ is a projective $B$-module. Hence, $B$ is smooth over $k$.        

\end{proof}

\begin{rem}
 We will prove next time that splitting and projective implies smoothness.
\end{rem}






\newpage
\section{Proofs of some statements from last time} 

\begin{prop}
	Let $A=k[x_{1},\ldots ,x_{n}]$, $B=A/I$. Then, $B$ is smooth over $k$ if (in fact if and only if) the conormal sequence
	\begin{equation}\label{sm41}
	 0 \rightarrow I/I^{2} \rightarrow \Omega _{A/k}\otimes B \rightarrow \Omega _{B/k} \rightarrow 0
	\end{equation}
is split exact.
\end{prop}
\begin{proof}
 Consider a square-zero extension 
 \begin{equation*}
  0 \rightarrow M \rightarrow E \rightarrow T \rightarrow 0
 \end{equation*}
 of $k$-algebras, and let $f:B \rightarrow T $ be a map of $k$-algebras. We will show that $f$ extends to a map $B \rightarrow E$. Note that we have a commutative diagram
 % https://q.uiver.app/#q=WzAsMTAsWzAsMCwiMCJdLFsyLDAsIlQiXSxbNCwwLCJBIl0sWzYsMCwiQiJdLFs4LDAsIjAiXSxbMiwyLCJNIl0sWzAsMiwiMCJdLFs0LDIsIkUiXSxbNiwyLCJUIl0sWzgsMiwiMCJdLFswLDFdLFsxLDJdLFsyLDNdLFszLDRdLFsxLDUsIiIsMCx7InN0eWxlIjp7ImJvZHkiOnsibmFtZSI6ImRhc2hlZCJ9fX1dLFs2LDVdLFs1LDddLFs3LDhdLFs4LDldLFszLDgsImYiXSxbMiw3LCJcXGV4aXN0cyBoIiwxLHsic3R5bGUiOnsiYm9keSI6eyJuYW1lIjoiZGFzaGVkIn19fV1d
\[\begin{tikzcd}
	0 && T && A && B && 0 \\
	\\
	0 && M && E && T && 0
	\arrow[from=1-1, to=1-3]
	\arrow[from=1-3, to=1-5]
	\arrow[dashed, from=1-3, to=3-3]
	\arrow[from=1-5, to=1-7]
	\arrow["{\exists h}"{description}, dashed, from=1-5, to=3-5]
	\arrow[from=1-7, to=1-9]
	\arrow["f", from=1-7, to=3-7]
	\arrow[from=3-1, to=3-3]
	\arrow[from=3-3, to=3-5]
	\arrow[from=3-5, to=3-7]
	\arrow[from=3-7, to=3-9]
\end{tikzcd}\]

As $A/k$ is smooth, there exists a map $h:A \rightarrow E $. The map $h|_{I}:I \rightarrow M $ induces a map $\overline{h}:I/I^{2} \rightarrow M  $ (as $M^{2}=0)$ Applying $\text{Hom}_{B}(\_,M)$ to \cref{sm41}, we get
\begin{equation*}
 0 \rightarrow \text{Hom}_{B}(\Omega _{B/k},M) \rightarrow \text{Hom}_{B}(\Omega _{A/k}\otimes_{A} B,M)=\text{Hom}_{k}(\Omega _{A/k},M) \rightarrow \text{Hom}_{B}(I/I^{2},M) \rightarrow 0    
\end{equation*}
Let $\theta $ be a lift of $\overline{h}\in \text{Hom}_{B}(I/I^{2},M)$. We may regard $\theta $ as a derivation $\tilde{\theta }\in \text{Der}_{k}(A,M)$. Check, as an exercise,  that $h'=h-\tilde{\theta }$ is a left inverse $h':B \rightarrow E$ of $f$.   
\end{proof}
\begin{prop}\label{propc}
	Suppose $A=k[x_{1},\ldots ,x_{n}]$, $B=A/I$. Then, for any $M\in B$-\emph{\textsf{Mod}}, there is an exact sequence 
\begin{equation*}
	0 \rightarrow T^{0}(B/k,M) \rightarrow \emph{\text{Hom}}(\Omega _{A/k},M) \rightarrow \emph{\text{Hom}}(I/I^{2},M) \rightarrow T^{1}(B/k,M) \rightarrow 0  
\end{equation*}
Further, $T^{2}(B/A,M)=T^{2}(B/k,M)$ 
\end{prop}
\begin{proof}
	We have maps of rings $k \rightarrow A \rightarrow B$. There is an exact sequence (by \cref{thmb}, known as the Jacobi Zariski Sequence) 

% https://q.uiver.app/#q=WzAsMTAsWzAsMCwiMCJdLFsxLDAsIlReMChCL0EsTSkiXSxbMiwwLCJUXjAoQi9rLE0pIl0sWzMsMCwiVF4wKEEvayxNKSJdLFsxLDEsIlReMShCL0EsTSkiXSxbMiwxLCJUXjEoQi9rLE0pIl0sWzMsMSwiVF4xKEEvayxNKSJdLFsxLDIsIlReMihCL0EsTSkiXSxbMiwyLCJUXjIoQi9rLE0pIl0sWzMsMiwiVF4yKEEvayxNKSJdLFswLDFdLFsxLDJdLFsyLDNdLFszLDRdLFs0LDVdLFs1LDZdLFs2LDddLFs3LDhdLFs4LDldXQ==
\[\begin{tikzcd}
	0 & {T^0(B/A,M)} & {T^0(B/k,M)} & {T^0(A/k,M)} \\
	& {T^1(B/A,M)} & {T^1(B/k,M)} & {T^1(A/k,M)} \\
	& {T^2(B/A,M)} & {T^2(B/k,M)} & {T^2(A/k,M)}
	\arrow[from=1-1, to=1-2]
	\arrow[from=1-2, to=1-3]
	\arrow[from=1-3, to=1-4]
	\arrow[from=1-4, to=2-2]
	\arrow[from=2-2, to=2-3]
	\arrow[from=2-3, to=2-4]
	\arrow[from=2-4, to=3-2]
	\arrow[from=3-2, to=3-3]
	\arrow[from=3-3, to=3-4]
\end{tikzcd}\]


The assertion follows from the following observations. 
\begin{enumerate}
 \item $T^{0}(B/A,M)=0$ since $A \twoheadrightarrow B$.
 \item  $T^{1}(B/A,M)=\text{Hom}_{B}(I/I^{2},M)$ by \cref{prop3}.
 \item $T^{0}(A/k,M)=\text{Hom}(\Omega _{A/k},M)$
 \item $T^{2}(A/k,M)=T^{1}(A/k,M)=0$ as $A$ is a polynomial ring over $k$.  
\end{enumerate}

\end{proof}
\begin{prop}\label{propd}
 Let $A$ be a local ring and $B=A/I$, where $I$ is generated by a regular sequence $a_{1},\ldots ,a_{n}$. Then, $T^{2}(B/A,M)=0$ for all $M$.    
\end{prop}
\begin{proof}
 Examine the construction of $\mathbb{L}_{\bullet}$ in this case.
 \begin{equation*}
  0 \rightarrow I \rightarrow A=R \rightarrow B \rightarrow 0
 \end{equation*}
 \begin{equation*}
	 0 \rightarrow Q \rightarrow F=A^{\oplus n} \xrightarrow{\theta } I \rightarrow 0 
 \end{equation*}

\end{proof}
\begin{prop}
 The construction of the $T^{i}$ functors is compatible with localisation. 
\end{prop}
\begin{proof}
 Left as an exercise.
\end{proof}
From last time,
\begin{thm}
Let $k$ be an algebraically closed field and $B$ a finite type $k$-algebra. Then, $B$ is smooth over $k$ if and only if $T^{1}(B/k,M)=0$ for all $M\in B$-\textsf{Mod}. Further, if $B$ is smooth over $k$, then also $T^{2}(B/k,M)=0$ for all $M\in B$-\textsf{Mod}.       
\end{thm}
\begin{proof}
	Suppose $B$ is smooth over $k$ and let $B = k[x_{1},\ldots ,x_{n}]/I$. By \cref{propc}, $T^{2}(B/k,M)=T^{2}(B/A,M)$. Consider the conormal sequence
	\begin{equation*}
	 0 \rightarrow I/I^{2} \rightarrow \Omega _{A/k} \otimes B \rightarrow \Omega _{B/k} \rightarrow 0
	\end{equation*}
Localise at any prime $\mathfrak{p} $ of $B$ 
\begin{equation*}
 0 \rightarrow (I/I^{2})_{\mathfrak{p} } \rightarrow \Omega _{A/k}\otimes B_{\mathfrak{p} } \rightarrow \Omega _{B/k, \mathfrak{p} } \rightarrow 0
\end{equation*}
We see that $I_{\mathfrak{p} }$ is generated by $n-1 = \text{dim}(A)-\text{dim}(B) $ elements in the regular local ring $A_{\mathfrak{p} }$. These generators form a regular sequence. This implies that $T^{2}(B_{\mathfrak{p} },M)$   
\end{proof}


\begin{thm}
	Let $A$ be a regular local $k$-algebra with residue field $k$ with $k=\overline{k} $ and let $B=A/I$. Then, $B$ is a local complete intersection in $A$ if and only if $T^{2}(B/k,M)=0$ for all $M\in B$-\emph{\textsf{Mod}}.        
\end{thm}
\begin{proof}
	As $A$ is regular and $k=\overline{k} $, $A$ is geometrically regular and hence, smooth over $k$. So $T^{i}(A/k,M)=0$ for all $M$ and for $i=1,2$. So, by the Jacobi-Zariski sequence (\cref{thmb}), get that $T^{2}(B/k,M)=T^{2}(B/A,M)$ for all $M$. If $B$ is a complete intersection in $A$, then localising, we get the vanishing $T^{2}(B_{\mathfrak{p} }/k, M) = 0$ for any prime $\mathfrak{p} $ of $B$. Conversely, suppose $T^{2}(B/k,M)=0$ for all $M$. As above, $T^{2}(B/A,M)=0$ for all $M$. We look at $\mathbb{L}_{\bullet}$.
	\begin{equation*}
	 0 \rightarrow I \rightarrow A \rightarrow B \rightarrow  0
	\end{equation*}
We may assume $F$  maps to a minimal set of generators of $I$.
\begin{equation*}
 0 \rightarrow Q \rightarrow F \rightarrow I \rightarrow 0
\end{equation*}
And,
\begin{equation*}
	\mathbb{L}_{\bullet}=\left( Q/F_{0} \xrightarrow{d_{2}} F/IF \xrightarrow{d_{1}} \Omega _{A/k} \otimes _{A} B \right) 
\end{equation*}
$T^{2}(B/A,M)=0$ implies that $\text{Hom}_{B}(F/IF,M) \rightarrow \text{Hom}_{B}(Q/F_{0},M)$ is surjective for all $M$. Take $M=Q/F_{0}$ and get a splitting $p:F/IF \rightarrow Q/F_{0} $ of $d_{2}$. Since $a_{1},\ldots ,a_{r}$ is a minimal set of generators of $I$, $Q\subseteq mF$. Thus, by Nakayama, $Q=F_{0}$ if and only if $H_{1}(K_{\bullet}(a_{1},\ldots ,a_{n}))$ is a regular sequence. Hence, $B$ is a local complete intersection.  

\end{proof}



\begin{comment}
Recollection (insert somewhere)
\begin{rem}
 $L_{2}=H_{1}(K(a_{1},\ldots ,a_{r}))$ 
\end{rem}
\end{comment}
\end{document}

