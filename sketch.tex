\documentclass{amsart}
\usepackage{capt-of}
\usepackage{xcolor}
\usepackage{fullpage}
\usepackage{amsmath}
\usepackage{amsfonts}
\usepackage{amssymb}
\usepackage{amsthm}
\usepackage{graphicx}
\usepackage{tikz-cd}
\usepackage{enumitem}
\usepackage{nicefrac}
\usepackage{hyperref}

\usepackage{cleveref}
\usepackage[mathscr]{euscript}
\usepackage{mlmodern}
\usepackage[T1]{fontenc}
\usepackage[]{mdframed}
\newtheorem{thm}{Theorem}
\newtheorem{cor}[thm]{Corollary}
\theoremstyle{definition}
\newtheorem{ex}[thm]{Example}

\newtheorem{lem}[thm]{Lemma}
\crefname{lem}{Lemma}{Lemmas}
\Crefname{lem}{Lemma}{Lemmas}
\newtheorem{defn}[thm]{Definition}
\newtheorem{notn}[thm]{Notation}
\newtheorem{conv}[thm]{Convention}
\theoremstyle{remark}
\newtheorem{rem}[thm]{Remark}
\newlist{myenum}{enumerate}{1}
\setlist[myenum,1]{label={(\arabic*)},
ref  ={(\arabic*)}}

\crefname{myenumi}{Remark}{Remarks}

\begin{document}
\begin{center}\textsc{\Large Hints and Solution Sketches to Problems in \cite{Har} \& Random Musings}
\end{center}
\vspace{0.5cm}










\begin{flushleft}
\textbf{Notation and Conventions:}
\begin{itemize}
	 \item All rings are assumed to be commutative with unity.
	 \item Given a topological space $X$, $\text{Top}(X)$ denotes the preorder category of open subsets, with the preorder given by inclusion.  
	\item \textsf{Set, Ring, Ab, Top, RS, LRS, Sch, AffSch, Var}$_k$ denote respectively the categories\footnote{The ``smallness assumption'' is suppressed throughout} of sets, (commutative) rings, Abelian groups, topological spaces, ringed spaces, locally ringed spaces, schemes, affines schemes and varieties over a field $k$. For a space $X$ and a bicomplete category $\mathscr{C}$, let $\textsf{Psh}_{\mathscr{C} }(X)$    and $\textsf{Sh}_{\mathscr{C} }(X)$ denote presheaves and sheaves over $X$ valued in $\mathscr{C}$ respectively. If $\mathscr{C}=\textsf{Set}$, we simply write $\textsf{Psh}(X)$ and $\textsf{Sh}(X)$ for $\textsf{Psh}_{\textsf{Set} }(X)$ and $\textsf{Sh}_{\textsf{Set} }(X)$
\end{itemize}
\end{flushleft}


\section*{Chapter II : Schemes}
\subsection*{Miscellaneous Remarks}
\begin{myenum}
\item \label{ppadj}\textbf{Pullback-Pushforward Adjunction}\\ 
\item \label{staskyadj}\textbf{Stalk-Skyscraper Adjunction}\\
Let $X\in \textsf{Top} $, $p\in X$ and $\mathscr{C} $ be a bicomplete category. Let $\text{St}_{p}: \textsf{Sh}_{\mathscr{C} }(X)\rightarrow \mathscr{C} $ denote the stalk (at $p$) functor and $\text{Sky}_{p}: \mathscr{C}  \rightarrow \textsf{Sh}_{\mathscr{C} }(X)$ the skyscraper. That is,    
\begin{equation*}
 (\text{Sky}_{p}a)(U):=\begin{cases}
	  a &\text{ if }p\in  U \\
	 \ast & \text{ if }p\notin U 
 	
 \end{cases}
\end{equation*}
Identifying $\mathscr{C} $ with $\textsf{Sh}_{\mathscr{C} }(\ast)$, the stalk and skyscraper functors are respectively the pullback and pushforward along $\{ p \} \hookrightarrow X$ and hence, $\text{St}_{p}: \textsf{Sh}_{\mathscr{C} }(X) \rightleftarrows \mathscr{C} : \text{Sky}_{p}$.  

\item \label{colring} \textbf{Abstract Nonsense: Colimits in }\textsf{Ring} \\
	The category of $\textsf{Ring}$ is both complete and cocomplete (see \cite[Chapter 3]{Bor2} for some generalities). We record a construction of colimits as this is useful, for instance, while computing fibre products of affine schemes and stalks of sheaves of rings.\\
	Let $I$ be a small category and $F:I \rightarrow \textsf{Ring}  $ be a functor. Let $G:=\bigsqcup_{i\in I} F(i)$ as sets. Define $\mathfrak{a}$ to be the ideal of $\mathbb{Z}[\{ x_{g}:g\in G \} ]$ generated by the union of the following collections of elements:
	\begin{enumerate}[label=(\roman*)]
	 \item $x_{a+b}-x_{a}-x_{b}$ for all $a,b\in F(i)$ and for all $i\in I$.
	 \item $x_{ab}-x_{a}x_{b}$ for all $a,b\in F(i)$ and for all $i\in I$.
	 \item $1_{\mathbb{Z}[\{ x_{g}:g\in G \} ]}-x_{1_{F(i)}}$ for all $i\in I$.
	 \item $x_{a}-x_{Ff(a)}$ for all $a\in F(i)$ and for all $f:i\rightarrow j$ in $\text{Mor}(I)$.    
	\end{enumerate}
	Let $R:=\mathbb{Z}[\{ x_{g}:g\in G \} ]/\mathfrak{a}$. By construction, the maps $F(i)\rightarrow R$, $a\mapsto x_{a}$ are ring maps and make $R$ a cocone of $F$. Further, for every other cocone $(R',\phi _{i}:F(i)\rightarrow R')$, the association $x_{a\in F(i)}\mapsto \phi_{i} (a)$ is the unique morphism of cocones from $R$ to $R'$. Thus, $R\cong \text{colim}_{I}F$.

\item \textbf{Abstract Nonsense: Limits in }\textsf{RS, LRS}\textbf{ and }\textsf{Sch}\\
\item \label{finbc} \textbf{(Locally of finite type/finite type/finite) morphisms are closed under base change}\\
 Suppose that $X\rightarrow Y$ is finite (resp. of finite type) and $Y'\rightarrow Y$ be a morphism of schemes. We claim that   $f':X \times_{Y} Y'\rightarrow Y'$, the base change along $g$, is finite (resp. of finite type) as well. 

Pick a point $y'\in Y'$ and let $y=g(y')$. Choose an affine open subscheme $y\in \text{Spec}(R_{y}) $ of $Y$ such that for a finite (resp. finitely generated) $R_{y}$-algebra $h:R_{y} \rightarrow S $, the following holds true.    


% https://q.uiver.app/#q=WzAsNSxbMSwyLCJcXHRleHR7U3BlY30oUl97eX0pIl0sWzMsMiwiWSJdLFszLDAsIlgiXSxbMSwwLCJYXFx0aW1lc19ZIFxcdGV4dHtTcGVjfShSX3t5fSkiXSxbMCwwLCJcXHRleHR7U3BlY30oUykiXSxbMCwxLCJnIl0sWzIsMSwiZiJdLFszLDAsImYnIiwyXSxbMywyXSxbMywxLCIiLDEseyJzdHlsZSI6eyJuYW1lIjoiY29ybmVyIn19XSxbNCwwLCJcXHRleHR7U3BlY30oaCkiLDJdLFs0LDMsIlxcY29uZyIsMSx7InN0eWxlIjp7ImJvZHkiOnsibmFtZSI6Im5vbmUifSwiaGVhZCI6eyJuYW1lIjoibm9uZSJ9fX1dXQ==
\[\begin{tikzcd}
	{\text{Spec}(S)} & {X\times_Y \text{Spec}(R_{y})} && X \\
	\\
	& {\text{Spec}(R_{y})} && Y
	\arrow["\cong"{description}, draw=none, from=1-1, to=1-2]
	\arrow["{\text{Spec}(h)}"', from=1-1, to=3-2]
	\arrow[from=1-2, to=1-4]
	\arrow["{f'}"', from=1-2, to=3-2]
	\arrow["\lrcorner"{anchor=center, pos=0.125}, draw=none, from=1-2, to=3-4]
	\arrow["f", from=1-4, to=3-4]
	\arrow["g", from=3-2, to=3-4]
\end{tikzcd}\]

Let $y'\in \text{Spec}(R_{y'}) $ be an affine open subscheme of $Y'$ contained in the open set $g^{-1}(\text{Spec}(R_{y}))$. Then, we have the following diagram where the dotted arrows are given by universality. 
% https://q.uiver.app/#q=WzAsOCxbMiw2LCJZJyJdLFs0LDQsIlkiXSxbNCwyLCJYIl0sWzEsMiwiXFx0ZXh0e1NwZWN9KFJfeSkiXSxbMSwwLCJcXHRleHR7U3BlYyhTKX0iXSxbMiw0LCJYXFx0aW1lc19ZIFknIl0sWzAsNCwiXFx0ZXh0e1NwZWN9KFJfe3knfSkiXSxbMCwyLCJcXHRleHR7U3BlY30oUl97eSd9KVxcdGltZXNfWSBYIl0sWzAsMSwiZyIsMV0sWzIsMSwiZiJdLFszLDEsIiIsMix7InN0eWxlIjp7InRhaWwiOnsibmFtZSI6Imhvb2siLCJzaWRlIjoidG9wIn19fV0sWzQsMl0sWzQsMywiXFx0ZXh0e1NwZWMoaCl9IiwxXSxbNSwwXSxbNSwyXSxbNiwwLCIiLDAseyJzdHlsZSI6eyJ0YWlsIjp7Im5hbWUiOiJob29rIiwic2lkZSI6InRvcCJ9fX1dLFs3LDQsIiIsMCx7InN0eWxlIjp7ImJvZHkiOnsibmFtZSI6ImRhc2hlZCJ9fX1dLFs3LDZdLFs3LDUsIiIsMSx7InN0eWxlIjp7ImJvZHkiOnsibmFtZSI6ImRhc2hlZCJ9fX1dLFs2LDMsImciLDEseyJsYWJlbF9wb3NpdGlvbiI6NzB9XSxbNSwxLCIiLDEseyJzdHlsZSI6eyJuYW1lIjoiY29ybmVyIn19XSxbNCwxLCIiLDEseyJzdHlsZSI6eyJuYW1lIjoiY29ybmVyIn19XV0=
\[\begin{tikzcd}
	& {\text{Spec(S)}} \\
	\\
	{\text{Spec}(R_{y'})\times_Y X} & {\text{Spec}(R_y)} &&& X \\
	\\
	{\text{Spec}(R_{y'})} && {X\times_Y Y'} && Y \\
	\\
	&& {Y'}
	\arrow["{\text{Spec(h)}}"{description}, from=1-2, to=3-2]
	\arrow[from=1-2, to=3-5]
	\arrow["\lrcorner"{anchor=center, pos=0.125}, draw=none, from=1-2, to=5-5]
	\arrow[dashed, from=3-1, to=1-2]
	\arrow[from=3-1, to=5-1]
	\arrow[dashed, from=3-1, to=5-3]
	\arrow[hook, from=3-2, to=5-5]
	\arrow["f", from=3-5, to=5-5]
	\arrow["g"{description}, from=5-1, to=3-2]
	\arrow[hook, from=5-1, to=7-3]
	\arrow[from=5-3, to=3-5]
	\arrow["\lrcorner"{anchor=center, pos=0.125, rotate=45}, draw=none, from=5-3, to=5-5]
	\arrow[from=5-3, to=7-3]
	\arrow["g"{description}, from=7-3, to=5-5]
\end{tikzcd}\]
Observe that by universality, the following subdiagrams are pullback squares too.
% https://q.uiver.app/#q=WzAsOCxbMCwwLCJcXHRleHR7U3BlY30oUl97eSd9KVxcdGltZXNfWSBYIl0sWzIsMCwiXFx0ZXh0e1NwZWN9KFMpIl0sWzIsMiwiXFx0ZXh0e1NwZWN9KFJfeSkiXSxbMCwyLCJcXHRleHR7U3BlY30oUl97eSd9KSJdLFs0LDAsIlxcdGV4dHtTcGVjfShSX3t5J30pXFx0aW1lc19ZIFgiXSxbNCwyLCJcXHRleHR7U3BlY30oUl97eSd9KSJdLFs2LDAsIlhcXHRpbWVzX1kgWSciXSxbNiwyLCJZJyJdLFswLDEsIiIsMCx7InN0eWxlIjp7ImJvZHkiOnsibmFtZSI6ImRhc2hlZCJ9fX1dLFsxLDJdLFszLDIsImciLDJdLFswLDNdLFswLDIsIiIsMSx7InN0eWxlIjp7Im5hbWUiOiJjb3JuZXIifX1dLFs0LDVdLFs0LDYsIiIsMix7InN0eWxlIjp7ImJvZHkiOnsibmFtZSI6ImRhc2hlZCJ9fX1dLFs2LDddLFs1LDcsIiIsMCx7InN0eWxlIjp7InRhaWwiOnsibmFtZSI6Imhvb2siLCJzaWRlIjoidG9wIn19fV0sWzQsNywiIiwxLHsic3R5bGUiOnsibmFtZSI6ImNvcm5lciJ9fV1d
\[\begin{tikzcd}
	{\text{Spec}(R_{y'})\times_Y X} && {\text{Spec}(S)} && {\text{Spec}(R_{y'})\times_Y X} && {X\times_Y Y'} \\
	\\
	{\text{Spec}(R_{y'})} && {\text{Spec}(R_y)} && {\text{Spec}(R_{y'})} && {Y'}
	\arrow[dashed, from=1-1, to=1-3]
	\arrow[from=1-1, to=3-1]
	\arrow["\lrcorner"{anchor=center, pos=0.125}, draw=none, from=1-1, to=3-3]
	\arrow[from=1-3, to=3-3]
	\arrow[dashed, from=1-5, to=1-7]
	\arrow[from=1-5, to=3-5]
	\arrow["\lrcorner"{anchor=center, pos=0.125}, draw=none, from=1-5, to=3-7]
	\arrow[from=1-7, to=3-7]
	\arrow["g"', from=3-1, to=3-3]
	\arrow[hook, from=3-5, to=3-7]
\end{tikzcd}\]
So, $\text{Spec}(R_{y'})\times_Y X \cong R_{y'}\otimes_{R_{y}} S$, which is a finite (resp. finitely generated) $R_{y'}$-algebra. 

\end{myenum}

\subsection*{Section 1. Sheaves} 


\begin{flushleft}
\textbf{Problem 1.3}
\end{flushleft}
\begin{enumerate}[label=(\alph*)]
 \item 
 \item Let $X$ be a topological space whose underlying set is $\{ a,b,c \} $ with $U:=\{ a,b \}$ $V:=\{ b,c \} $ and $U\cap V$ exactly being the proper non-trivial open subsets. Then, the obvious map $$\text{Hom}_{\text{Top}(X)}(\_,U)\bigsqcup \text{Hom}_{\text{Top}(X)}(\_,V)\rightarrow \text{Hom}_{\text{Top}(X)}(\_,X)$$ is surjective at the level of stalks but not on global sections. Note that contravariant representable functors are sheaves and sheafification, being a left adjoint, preserves colimits. Finally, apply the free functor if one instead wants a sheaf valued in $\textsf{Ring}$, $R$-$\textsf{Mod}$ etc.    
\end{enumerate}


\begin{flushleft}
 \textbf{Problem 1.17}\quad 
See \cref{staskyadj}.
\end{flushleft}
\begin{flushleft}
	\textbf{Problem 1.18}\quad See \cref{ppadj} 
\end{flushleft}

\subsection*{Section 2. Schemes} 
\begin{flushleft}
 \textbf{Problem 2.2} 
\end{flushleft}
Straightforward. Follows from the fact that distinguished open subsets of affine schemes with the restricted structure sheaf are again affine.

\begin{flushleft}
 \textbf{Problem 2.3} (Reduced Schemes)

\end{flushleft}
 \begin{enumerate}[label=(\alph*)]
 \item The sheaf condition in terms of the equaliser implies that a scheme $(X,\mathcal{O}_{X})$ is reduced if and only if all its affine open subschemes are reduced. Hence, consider $\text{Spec}(R)$ for a ring $R$. If there is a non-zero nilpotent $(x_{\mathfrak{p} })_{\mathfrak{p}\in U}$  in $\mathcal{O}_{\text{Spec}(R)}(U)$ for some open $U\subseteq \text{Spec}(R)$, then for some $\mathfrak{p}\in U$, $x_{\mathfrak{p} }$ is non-zero and hence, a nilpotent in $R_{\mathfrak{p} }$. Conversely suppose that the stalk at some $\mathfrak{p}\in \text{Spec}(R) $ (which is $R_{\mathfrak{p} }$)  has a non-zero nilpotent. Then, $R$  has a non-zero nilpotent and subsequently so does the ring of global sections.
 \item 
\end{enumerate}



\subsection*{Section 3. First Properties of Schemes} 
\begin{flushleft}
 \textbf{Problem 3.1} 
\end{flushleft}
\begin{enumerate}
 \item One direction is obvious. For the other, we have the adjunction $\textsf{LRS} : \Gamma(\cdot,\mathcal{O}_{\cdot})  \rightleftarrows \text{Spec} :\textsf{Ring}^{\text{op}}$ which is an equivalence upon restriction to affine schemes. Hence, for a ring $R$ and non-zero $f\in R$, $R \hookrightarrow R_{f}$ corresponds (up to isomorphism) to the inclusion $D_{f}\hookrightarrow \text{Spec}(R)$.
 \item Cover a given affine open subset with finitely many (by compactness) ``nice'' affine open subsets which exist by the locally finite type assumption. Work with distinguished open subsets on both sides using (1).
\end{enumerate}

\begin{flushleft}
	\textbf{Problem 3.2} \label{p3.2}
\end{flushleft}



\begin{flushleft}
	\textbf{Problem 3.3} \label{p3.3}
\end{flushleft}

\begin{flushleft}
	\textbf{Problem 3.4}  \label{p3.4}
\end{flushleft}

\begin{flushleft}
 \textbf{Problem 3.9} 
\end{flushleft}
\begin{enumerate}[label=(\alph*)]
	\item $\text{Spec}$ is a right adjoint. So, $$\text{Spec}(k[x])\times_{\text{Spec}(k)} \text{Spec}(k[x])\cong \text{Spec}(k[x] \bigsqcup_{k} k[x])\cong \text{Spec}(k[x]\otimes_{k} k[x])\cong \text{Spec}(k[x,y])$$
The underlying point sets are the union of two intersecting lines and the plane and hence, distinct.
\item Similar to (a), the fibred product is $\text{Spec}(k(s)\otimes_{k} k(t))$. Letting $X:=\{ fg\in k[s,t]|f\in k[s]\setminus 0 \text{ and }f\in k[t]\setminus 0 \} $, $k(s)\otimes_{k} k(t) \cong X^{-1}k[s,t] $. This is clearly a domain but not a field and hence its spectrum contains more than one point.
\end{enumerate}

\begin{flushleft}
\textbf{Problem 3.13}
\end{flushleft}
\begin{enumerate}[label=(\alph*)]
 \item 
 \item 
 \item 
 \item 
 \item
 \item 
 \item By Hilbert basis theorem and since quotients of Noetherian rings are Noetherian, if $Y$ is Noetherian and $X$ is of finite type, then $X$ is locally Noetherian. By \hyperref[p3.2]{Problem 3.2} and \hyperref[p3.3]{Problem 3.3 (a)}, $X$ is also quasicompact. 
\end{enumerate}



\begin{flushleft}
 \textbf{Problem 3.14} 
\end{flushleft}
(Pedantic Remark: by saying ``closed points are dense'', Hartshorne means that the set of closed points is dense)\\
By \hyperref[p3.4]{Problem 3.4}, we may assume WLOG that $X=\text{Spec}(R)$ for a finite $R$-algebra $k$. Since $k$ is algebraically closed, this means that $k=R$.\\
Take ??? 





\begin{flushleft}
 \textbf{Problem 3.16} 
\end{flushleft}
Vacuously, $\emptyset$ satisfies $\mathscr{P}$. Suppose towards contradiction that $X_{0}:=X$ does not. Then, one can find $\emptyset \subsetneq X_{1} \subsetneq X$ that does not satisfy $\mathscr{P} $. Suppose for $1\leqslant i\leqslant k$,  $X_{i}\subsetneq X_{i-1}$ and $X_{i}$ does not satisfy $\mathscr{P} $. Then, there exists $\emptyset \subsetneq X_{i+1} \subsetneq X_{i}$ not satisfying $\mathscr{P} $. By induction, this implies that $X$ is not Noetherian.    

\begin{flushleft}
 \textbf{Problem 3.17} 
\end{flushleft}
\begin{enumerate}[label=(\alph*)]
\item The generic point condition follows from \hyperref[p2.9]{Problem 2.9}. Write $X$ as a finite union of $\text{Spec}(R_{i})$ with $R_{i}$ being Noetherian and hence, $\text{Spec}(R_{i})$ being Noetherian as well. Suppose towards contradiction that there exists a strictly decreasing chain of closed subsets of $X$. Take intersections with the complements $\text{Spec}(R_{i})^{\text{c}}$. All of these stabilise and these complements cover $X$, leading to the required contradiction.
 \item A minimal closed subset $A=\bigcap\limits_{a\in A} \overline{a}$. If $A$ contains more than one element, then it is irreducible with more than one generic point.
 \item Given $x\in X$, the intersection of all closed subsets containing $x$ is a minimal non-empty closed subset of $X$. $T_{0}$ easily follows.
 \item Trivial, even to me.
\end{enumerate}

\subsection*{4. Separated and Proper Morphisms} 
\begin{flushleft}
 \textbf{Problem 4.1} 
\end{flushleft}




By \hyperref[p3.5]{Problem 3.5 (b)} and \cref{finbc}, finite morphisms are universally closed. Finite morphisms are separated by \cite[Chapter 2, Proposition 4.1, Corollary 4.6 (f)]{Har}.



\subsection*{5. Sheaves of Modules}
\begin{flushleft}
	\textbf{Problem 5.8} \label{p5.8}
\end{flushleft}

\begin{thebibliography}{Neil1234}
	\bibitem[Bor95]{Bor2} Francis Borceaux. Handbook of Categorical Algebra, Volume 2: Categories and Structures. 
	\bibitem[Har10]{Har} Robin Hartshorne. \emph{Algebraic Geometry}.
\end{thebibliography}
\end{document}
