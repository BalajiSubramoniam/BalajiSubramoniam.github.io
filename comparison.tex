\documentclass[10pt]{amsart}
\usepackage{capt-of}
\usepackage{xcolor}
\usepackage{amsmath}
\usepackage{amsfonts}
\usepackage{amssymb}
\usepackage[margin=1.5cm]{geometry}
\usepackage{graphicx}
\usepackage{tikz-cd}
\usepackage{enumitem}
\usepackage{nicefrac}
\usepackage{hyperref}
\usepackage{mathrsfs}
\usepackage{mlmodern}
\usepackage[capitalise]{cleveref}
\usepackage{amsthm}
\usepackage[T1]{fontenc}
\usepackage[]{mdframed}
\setcounter{section}{-1}
\setcounter{tocdepth}{1}
\usepackage{etoolbox}
\theoremstyle{definition}
\newtheorem{thm}{Theorem}
\newtheorem{prop}[thm]{Proposition}
\crefname{prop}{Proposition}{Propositions}
\newtheorem{cor}[thm]{Corollary}
\newtheorem{ex}[thm]{Example}
\newtheorem{defn}[thm]{Definition}
\newtheorem{notn}[thm]{Notation}
\newtheorem{conv}[thm]{Convention}
\newtheorem{rem}[thm]{Remark}
\begin{document}
\thispagestyle{empty}
\begin{center}
 \textbf{\large Milnor's Definition of a Manifold} 
\end{center}
\vspace{0.2cm}
The aim of this document is to compare the definitions of an $n$-manifold as in \cite{Mil} and \cite{Lee}. For convenience, we shall refer to smooth $n$-manifolds according to \cite{Mil} ``Milnor $n$-manifolds''. What we call a ``chart $n$-manifold'' is precisely a smooth $n$-manifolds according to, for example, \cite{Lee} \emph{except} that we do not assume the underlying topological space to be second countable.

\begin{conv}
 $\mathbb{R} ^{n}$ will be viewed both as a chart $n$-manifold and a Milnor $n$-manifold with the singleton chart and the singleton local parametrisation respectively.
\end{conv}
\begin{notn}
 For topological spaces $A$ and $B$, ``$A\subseteq_{\text{o} } B$'' means ``$A$ is an open subspace of $B$''.  
\end{notn}

\begin{notn}
 Given a chart $n$-manifold $M$, let $F_{M}$ denote the set of smooth (in the ``chart sense'') real valued functions on $M$. Let $i : M \rightarrow \mathbb{R} ^{F_{M}} $ denote the evaluation map. 
\end{notn}

\begin{prop}\label{mainprop}
	Let $M$ be a chart $n$-manifold. Then, $i(M)$ is canonically a Milnor manifold.  
\end{prop}
\begin{proof}
 Fix $x\in M$.  Let $\alpha :\mathbb{R} ^{n} \rightarrow U \subseteq_{\text{o} } M$ be a chart around $x$ in the maximal atlas of $M$. We claim that the composition
 \begin{align*}
	 \beta :\mathbb{R} ^{n}\xrightarrow{\alpha } U \xrightarrow{i} i(U)\subseteq \mathbb{R} ^{F_{M}}
 \end{align*}
 is a local parametrisation around $i(x)$. In order to prove this, it is sufficient to show the following:
 \begin{enumerate}[label=(\arabic*)]
  \item $\beta $ is smooth.
\item For all $y\in \mathbb{R} ^{n}$, $D\beta \vert_{y}$ has rank $n$.   
 \item $\text{Im}(\beta )$ is open and $\beta$ is a homeomorphism onto its image.
 \end{enumerate}
 For $f\in F_{M}$, let $\pi _{f}:\mathbb{R} ^{F_{M}}\rightarrow \mathbb{R} $ denote the projection to the $f^{\text{th} }$ coordinate. Then, for any $f\in F_{M}$, $\pi _{f}\circ \beta = f\vert_{U}\circ \alpha $, which is smooth. Hence, $\beta $ is smooth.

 For $i=1,\ldots ,n$, let $\psi _{i}:U \rightarrow \mathbb{R}  $ such that $\psi_i \circ \alpha =\pi _{i}$. Clearly, $\psi_{i}$ is smooth on $U$. Fix $y\in \mathbb{R} ^{n}$. Let $g_{i}\in F_{M}$ such that $g_{i}$ agrees with $\psi _{i}$ in an open neighbourhood of $\alpha (y)$. Consider the composition
 \begin{align*}
	 \mathbb{R} ^{n}\xrightarrow{\beta } i(M)\subseteq \mathbb{R} ^{F_{M}}\xrightarrow{\pi _{g_{1}},\ldots ,\pi _{g_{n}}} \mathbb{R} ^{n} 
 \end{align*}
 whose components are clearly $g_{i}\circ \alpha $. Thus,
 \begin{align*}
  \frac{\partial}{\partial u_{j}}(g_{i}\circ \alpha )\bigg\vert_{y}= \frac{\partial}{\partial u_{j}}(\psi _{i}\circ \alpha )\bigg\vert_{y} = \frac{\partial u_{i}}{\partial u_{j}}\bigg\vert_{y}=\begin{cases}
  	1 &\text{ if }i=j\\
	0 & \text{ otherwise } 
  \end{cases}
 \end{align*}
 This proves (2). Lastly, suppose that $V\subseteq_{\text{o} } U$. For all $v\in V$, let $g_{v}\in F_{M}$ such that $g_{v}(v)>0$ and $\text{supp}(g_{v})\subseteq V$. Let $\pi _{g_{v}}:\mathbb{R} ^{F_{M}} \rightarrow \mathbb{R}  $ denote the projection to the $g_{v}^{\text{th} }$ coordinate. Then, 
 \begin{itemize}
  \item $\pi _{g_{v}}^{-1}(\mathbb{R} \setminus \{ 0 \} )\subseteq_{\text{o} } \mathbb{R} ^{F_{M}}$.
 \item $i(v)\in \pi _{g_{v}}^{-1}(\mathbb{R} \setminus \{ 0 \} ) \cap i(M)\subseteq _{\text{o} } i(M)$.
 \item $i(m)\notin \pi _{g_{v}}^{-1}(\mathbb{R} \setminus \{ 0 \} )\cap i(M)$ for all $m\in M\setminus V$.
 \end{itemize}
Hence, 
\begin{align*}
 i(V)=\bigcup\limits_{v\in V} \pi ^{-1}_{g_{v}} (\mathbb{R} \setminus \{ 0 \} )\cap i(M) \subseteq _{\text{o} } i(M) 
\end{align*}

This proves (3).
\end{proof}
%\begin{rem}
%If $\pi _{0}:\mathbb{R} ^{F_{M}} \rightarrow \mathbb{R}  $ denotes the projection to the coordinate corresponding to the constant function at $0$, then $\pi ^{-1}(0)\subseteq_{\text{c} } \mathbb{R} ^{F_{M}}$ and hence, $M\rightarrow \mathbb{R} ^{F_{M}}$ is a topological embedding.     
%\end{rem}

\begin{rem}
	In the proof of \cref{mainprop}, we might as well have relaxed the assumption that the domain of $\alpha $ is $\mathbb{R} ^{n}$ and the same argument could have been used to construct a local parametrisation. The constructed Milnor manifold structure is said to be canonical as any chart translates to a local parametrisation by pre-composition.  
\end{rem}
\begin{rem}
	\cref{mainprop} essentially says that a chart $n$-manifold admits a smooth embedding into $\mathbb{R} ^{A}$ for some set $A$.     
\end{rem}

\begin{rem}
	Conversely, given a Milnor $n$-manifold $N$, the local parametrisations on the underlying topological space provide an atlas for $N$, making it a chart $n$-manifold. It is also easy to see, under this correspondence, that the definitions of smooth functions (see \cite[P.17]{Mil} and \cite[P.34]{Lee}) coincide.
\end{rem}


\begin{thebibliography}{Neil1234}
	\bibitem[MS74]{Mil} John Milnor and James Stasheff. \emph{Characteristic Classes}.
	\bibitem[Lee00]{Lee} John Lee. \emph{Introduction to Smooth Manifolds}.
\end{thebibliography}

\end{document}
